\documentclass{pretexto/report}
% --- Archivo de bibliografía -----------------
\addbibresource{pretexto/repbib.bib}


\title{Análisis y diseño}

%%%%%%%%%%%%%%%%%%%% TERMINA PREÁMBULO %%%%%%%%%%%%

\begin{document}

%%%%%%%%%%%%%%%%%%%% PORTADA %%%%%%%%%%%%%%%%%%%%%%%

\include{pretexto/portada.tex}                                                     

%%%%%%%%%%%%%%%%%%%% ÍNDICES %%%%%%%%%%%%%%%%%%%%%%%%%%
\tableofcontents 
\pagebreak


%%%%%%%%%%%%%%%%% INTRODUCCIÓN %%%%%%%%%%%%%%%%%%%%%%%%

\pagebreak
\section{Análisis y Diseño del sistema}

\subsection{Objetivos}

\begin{itemize}
    \item Desarrollar un sistema que permita al usuario registrar sus habitos y tareas diarias
    \item Añadir espacios para que el usuario agregue datos que sirvan de contexto para las recomendaciones inteligentes.
    \item Implementar un modulo que realice recomendaciones inteligentes sobre como realizar los habitos y tareas registradas
    \item Realizar una interfaz intuitiva y atractiva que permita añadir las tareas y habitos de manera facil
\end{itemize}


\subsection{Diagrama de Entradas y Salidas}
\begin{figure}[H]
    \centering
    \includegraphics[width=\linewidth]{pdfs/e_s.pdf}
    \caption{Diagrama generado con Mermaid}
\end{figure}

\subsection{Diagrama Entidad-Relación}
\begin{figure}[H]
    \centering
    \includegraphics[width=\linewidth]{pdfs/relacional.pdf}
    \caption{Diagrama generado con Mermaid}
\end{figure}

\subsection{Diagrama de Procesos}
\begin{figure}[H]
    \centering
    \includegraphics[width=0.8]{pdfs/procesos-2-3.pdf}
    \caption{Diagrama generado con Mermaid}
\end{figure}

# TODO: Quitar las hojas en blanco entre diagramas
# TODO: Poner hojas en horizontal para los diagramas horizontalmente grandes
# TODO: separar el diagrama de casos de uso en use_Cases_admin y use_cases_usuario


\subsection{Diagrama de Casos de Uso}
\begin{figure}[H]
    \centering
    \includegraphics[width=\linewidth]{pdfs/use_cases.pdf}
    \caption{Diagrama generado con Mermaid}
\end{figure}


%mmdc -i diagramas/nombre_archivo.mmd -o pdfs/nombre_archivo.pdf -b transparent -t default

%%%%%%%%%% BIBLIOGRAFÍA %%%%%%%%%%%%%%%%%%%%%
\pagebreak
\printbibliography[heading=bibintoc]

\end{document}
